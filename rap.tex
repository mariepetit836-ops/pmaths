\documentclass{article}
%graphique
\usepackage{epsfig,graphicx,float}
%math
\usepackage{amsmath}

\newcommand{\be}{\begin{equation}}
\newcommand{\ee}{\end{equation}}
\usepackage{graphicx}

%----------NE PAS MODIFIER CE QUI EST AU DESSUS-------------

\begin{document}


\author{D\'eborah Bernon, Alp Ege Kurtulus, Marie Petit, Yohann Piouceau}
\title{It\'erations non-lin\'eaires : bifurcation et chaos}

\date{\today}
\maketitle

\section*{Synth\`ese du projet}
Nous avons mod\'elise l'\'evolution d'une population dans le temps
en examinant comment la solution varie lorsque l'on change les param\`etres.

\tableofcontents

\section{TODO}

\begin{itemize}
	\item comprendre les cycles 2, 4 ...
	\item diagramme bifurcation $(a,x_\infty)$: code et resultats
	\item histogramme


\end{itemize}


\section{Introduction}
Les buts du projet sont d'analyser l'it\'eration logistique ci-dessous, de comprendre l'\'evolution des solutions et de voir le chaos :
$$f(x)=ax( 1 - x )$$ si$$0 \leq a < {4}$$
On examinera de la m\^eme facon les fonctions :
$$f(x)=ax \pm x^3$$

On suivra le plan d'avancement suivant : \\
- Expliquer $x_{n+1}=ax_n( 1 - x_n )$ en supposant $x_n$ une population au jour $n$ \\
- Examiner les points fixes de l'it\'eration, sont-ils stables ? \\
- Examiner les points fixes en fonction de $a$. Que se passent-ils lorsqu'ils se d\'estabilisent \\
- Faire un code num\'erique logi.pas \\
- Faire un sch\'ema de bifurcation o\`u on examinera les solutions en fonction de $a$\\
- Que se passe-t-il pour les grandes valeurs de $a$ ? Faire un histogramme des solutions dans les r\'egions chaotiques \\








\section{Etude mathématique}

Dans un premier temps, on va expliquer le mod\`ele $x_{n+1}=ax_n( 1 - x_n )$ pour l'\'evolution d'une population.

Le premier terme : croissance de la population (croissance exponentielle).
Code permettant d'obtenir les diagrammes d'evolution de la suite en fonction de $n$, suivant les valeurs de $a$. \\$->$ plus $a$ est grand et plus les bact\'eries se multiplient/ d\'eveloppent rapidement.

Le second terme : facteur de r\'egulation/ limitation de la population. \\$->$ quand $x_n$ tend vers $0$, la croissance est maximale pour une valeur de $a$ et dans le cas contraire quand $x_n$ tend vers $1$, la croissance s'arr\^ete compl\`etement.

La combinaison de ces termes donnent une mod\'elisation de population qui cro\^it exponentiellement au d\'ebut, puis qui ralentit \`a mesure que la population approche de la capacit\'e maximale du milieu (emp\^echant une croissance infinie).

Dans un second temps, on examine les points fixes de l'it\'eration et leur stabilit\'e.


Recherche des points fixes :

  On pose x* un point fixe, c'est \`a dire $$x* = f(x*)$$
  R\'esolution de l'\'equation : $$x* = ax (1 - x*) $$
  D'o\`u $$x_1 = 0$$ et $$x_2 = 1 - 1/a$$

$x_1$ est valable quand $a > 1$, autrement $f(x)$ tend vers $0$.

La stabilit\'e v\'erifie la condition suivante : 
Si le point fixe $x*$ est stable alors 
$$|f'(x*)| < 1$$
On a $f'(x) = - 2ax + a$, évaluons en $x_2 = 0$, on a : $$f'(0) = a$$ et en $x_1 = 1 - 1/a$, on a : $$f'(x_2) = - a + 2$$ et $$ - 1 < 2 - a < 1$$ donc $$ 1 < a \leq 3$$
Ainsi, $x*$ (point fixe) est stable si $a \in ]1, 3]$ (la population se stabilise) ou si $a \in [0, 1[$ (la population s'\'eteint) et pour $ a < 3$ c'est instable et on voit un ph\'enom\`eme de bifurcation qui m\`ene au chaos.\\


\section{Etude informatique}

On cr\'ee un programme logi.pas qui permet de visualiser la fonction $f(x)=ax( 1 - x )$ en modifiant, dans un fichier d1 ou d2, les valeurs de $a$, de $x$ et de $n$. C'est-\`a-dire qu'elle permet de voir la croissance de la population selon le coefficient $a$ et le facteur de régulation $x$ pour une dur\'ee de $n$ jours. //

Pour les references on utilise cite, par exemple \cite{crs01}. Les references
sont mises avec des bibitem a la fin du fichier.

\begin{verbatim}

#include<stdlib.h>
#include<stdio.h>
#include<math.h>



int main()
{
int jk, k;
float a, p0, p1, var;
		scanf("%d %f %f",&jk, &p0, &a);
	for( k = 1 ; k <= jk ; k++)
	{
		p1 = a*p0*( 1 - p0 );
		var = ( ( p1 - p0) / p0 )*100;
		printf("%d  %f %f \n",k, p1, var );
		p0 = p1;

	}
}


\end{verbatim}


Code permettant d'obtenir le diagramme de bifurcation, c'est a dire celui donnant les valeurs des points des points fixes en fonction de a.

     Expression initiale : $$x_{n+1} = ax_n(1 - x_n)$$ pour  $$0 < a < 4$$
  c'est a dire $$x_{n+1} = f(x_n) = ax_n(1 - x_n)$$

On peut inserer les fichiers graphiques postcript (.eps ou .ps) avec
epsfig voir Fig. \ref{fig1}. Les autres formats graphiques .jpg .png
peuvent etre inseres avec includegraphics  voir Fig. \ref{fig2}.

%Insertion d'une figure en eps dans le texte avec epsfig
\begin{figure} [H] \label{fig1}
\centerline{\epsfig{file=ham.eps,height=12cm,width=24 cm}}
\caption{Facon avec epsfig}
\end{figure}

\begin{figure} [H]
\centering
\resizebox{14 cm}{5 cm}{\includegraphics{ham}}
  \caption{Une facon avec includegraphics}
  \label{fig2}
\end{figure}





\begin{thebibliography}{99}


\bibitem{crs01} D. Cvetkovic, P. Rowlinson and S. Simic,
 "An Introduction to the Theory of
Graph Spectra",  London Mathematical Society Student Texts (No. 75), (2001).
\end{thebibliography}


\end{document}
