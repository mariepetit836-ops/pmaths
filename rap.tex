\documentclass{article}
%graphique
\usepackage{epsfig,graphicx,float}
%math
\usepackage{amsmath}
\usepackage{hyperref}

\newcommand{\be}{\begin{equation}}
\newcommand{\ee}{\end{equation}}
\usepackage{graphicx}

%----------NE PAS MODIFIER CE QUI EST AU DESSUS-------------

\begin{document}


\author{D\'eborah Bernon, Alp Ege Kurtulus, Marie Petit, Yohann Piouceau}
\title{It\'erations non-lin\'eaires : bifurcation et chaos}

\date{\today}
\maketitle

\section*{Synth\`ese du projet}
Nous avons mod\'elise l'\'evolution d'une population dans le temps
en examinant comment la solution varie lorsque l'on change les param\`etres.

\tableofcontents


\section{Introduction}
Les buts du projet sont d'analyser l'it\'eration logistique ci-dessous, de comprendre l'\'evolution des solutions et de voir le chaos :
$$f(x)=ax( 1 - x )$$ si$$0 \leq a < {4}$$
\\

On suivra le plan d'avancement suivant : \\
- Expliquer $x_{n+1}=ax_n( 1 - x_n )$ en supposant $x_n$ une population au jour $n$ \\
- Examiner les points fixes de l'it\'eration, sont-ils stables ? \\
- Examiner les points fixes en fonction de $a$. Que se passent-ils lorsqu'ils se d\'estabilisent \\
- Faire un code num\'erique logi.pas \\
- Faire un sch\'ema de bifurcation o\`u on examinera les solutions en fonction de $a$\\
- Que se passe-t-il pour les grandes valeurs de $a$ ? \\








\section{Etude mathématique}

Dans un premier temps, on va expliquer le mod\`ele $x_{n+1}=ax_n( 1 - x_n )$ pour l'\'evolution d'une population.

Le premier terme : croissance de la population (croissance exponentielle).
Code permettant d'obtenir les diagrammes d'evolution de la suite en fonction de $n$, suivant les valeurs de $a$. \\$->$ plus $a$ est grand et plus les bact\'eries se multiplient/ d\'eveloppent rapidement.

Le second terme : facteur de r\'egulation/ limitation de la population. \\$->$ quand $x_n$ tend vers $0$, la croissance est maximale pour une valeur de $a$ et dans le cas contraire quand $x_n$ tend vers $1$, la croissance s'arr\^ete compl\`etement.

La combinaison de ces termes donnent une mod\'elisation de population qui cro\^it exponentiellement au d\'ebut, puis qui ralentit \`a mesure que la population approche de la capacit\'e maximale du milieu (emp\^echant une croissance infinie).

Dans un second temps, on examine les points fixes de l'it\'eration et leur stabilit\'e.


Recherche des points fixes :

  On pose x* un point fixe, c'est \`a dire $$x* = f(x*)$$
  R\'esolution de l'\'equation : $$x* = ax (1 - x*) $$
  D'o\`u $$x_1 = 0$$ et $$x_2 = 1 - \frac{1}{a}$$

$x_2$ est valable quand $a > 1$, autrement $f(x)$ tend vers $0$.

La stabilit\'e v\'erifie la condition suivante : 
Si le point fixe $x*$ est stable alors 
$$|f'(x*)| < 1$$
Concrètement, la pente de la courbe doit être inférieure à $y=x$.

\begin{figure} [H] \label{fig1}
\centerline{\epsfig{file=Graph_of_logistic_map.png,height=5cm,width=6cm}}
\caption{Diagramme de la suite logistique}
\end{figure}

Tant que le point fixe est stable, le système continue de suivre ce modèle mais lorsque la pente augmente, la déviation s'amplifie et c'est ici que le phénomène de bifurcation né.\\

On a $f'(x) = - 2ax + a$, évaluons en $x_2 = 0$, on a : $$f'(0) = a$$ et en $x_1 = 1 - \frac{1}{a}$, on a : $$f'(x_1) = - a + 2$$ et $$ - 1 < 2 - a < 1$$ donc $$ 1 < a \leq 3$$
Ainsi, $x*$ (point fixe) est stable si $a \in ]1, 3]$ (la population se stabilise) ou si $a \in [0, 1[$ (la population s'\'eteint) et pour $ a > 3$ c'est instable et on voit un ph\'enom\`eme de bifurcation qui m\`ene au chaos.\\

De façon graphique, on observe que, pour $a\geq3$, la fonction oscille entre 2 points. Pour trouver ces points, il faut déterminer les points fixes de la fonction $f\circ f$. Ils correspondent aux cycles de période 2 de la fonction logistique.\\

On fait donc $f\circ f$ :
$$f\circ f=a(ax_n(1-x_n))(1-(ax_n(1-x_n)))$$
On obtient $$f\circ f=a^2x_n(1-x_n)-a^3(x_n)^2(1-x_n)^2$$
On cherche la d\'eriv\'ee de $f\circ f$:
$$(f\circ f)'=a^2(1-2x_n+a2x_n+2a(x_n)^2)$$
En cherchant les points fixes de $(f\circ f)'$, on a :\\
\\
$x_3=\frac{(a+1)-\sqrt{(a+1)(a-3)}}{2a}$ et $x_4=\frac{(a+1)+\sqrt{(a+1)(a-3)}}{2a}$\\
\\
Ces cycles apparaissent au moment de la première bifurcation (vers $a = 3$). Une cascade de bifurcations successives mène ensuite au chaos.\\




\section{Etude informatique}

On cr\'ee un programme logi.pas qui permet de visualiser la fonction $f(x)=ax( 1 - x )$ en modifiant, dans un fichier d1 ou d2, les valeurs de $a$, de $x$ et de $n$. C'est-\`a-dire qu'elle permet de voir la croissance de la population selon le coefficient $a$ et le facteur de régulation $x$ pour une dur\'ee de $n$ jours. \\

On cr\'ee \'egalement un programme alogi.pas pour afficher le diagramme de bifurcation, qui trace $x$ en fonction de $a$, et pour cela on utilise le code ci-dessous:

\begin{verbatim}
program alogi;

var x, amax, amin, da, a : Real;
	i, j, k, na, n1, n2 : Word;
begin
	read(x, n1, n2, amax, amin, na);
	da:=(amax-amin)/(na-1);
	for i:=1 to na do
	begin
		a:=amin+(i-1)*da;
		for j:=1 to n1 do
		x:=a*x*(1-x);
		for k:=1 to n2 do
		begin
			x:=a*x*(1-x);
			writeln(a:2:2,'       ',x:1:5);
		end;
	end
end.

\end{verbatim}


Code permettant d'obtenir le diagramme de bifurcation, c'est-\`a-dire celui donnant les valeurs des points fixes en fonction de $a$. On a donc $x$ sur l'axe des ordonnées et $a$ sur l'axe des abscisses.\\

On obtient les diagrammes ci-dessous. Pour les deux, on a trac\'e les valeurs de $x$ à l'infini (réellement à 4000) en fonction de $a$, qu'on a fait varier entre deux bornes. La Figure 2 représente une intervalle de $a$ de 0.5 à 4, qui nous donne une image globale. Puis, on a augmenté la borne inférieure de $a$ à 2.5 qui est plus proche à 3, et la Figure 3 nous permet donc d'observer la bifurcation en détail.\\


\begin{figure} [H] \label{fig2}
\centerline{\epsfig{file=asimple.eps,height=6cm,width=12cm}}
\caption{diagramme de bifurcation 1}
\end{figure}


\begin{figure} [H] \label{fig3}
\centerline{\epsfig{file=a1simple.eps,height=6cm,width=12cm}}
\caption{diagramme de bifurcation 2}
\end{figure}

Les figures obtenues montrent clairement la zone stable pour $a<3$, les bifurcations successives, ainsi que la zone chaotique.


\section{Conclusion}

Ce projet a permis d’illustrer comment un modèle simple peut produire une grande variété de comportements dynamiques. L’étude mathématique met en évidence les points fixes et leurs domaines de stabilité. L’implémentation informatique, notamment via le diagramme de bifurcation, montre l’apparition des cycles et du chaos.

Ainsi, l’itération logistique constitue un exemple saisissant de la manière dont des règles déterministes peuvent engendrer une dynamique imprévisible, marquant la frontière entre stabilité, périodicité et chaos.\\

\section{Sources}

\begin{itemize}
	\item\url{https://en.wikipedia.org/wiki/Logistic_map}
	\item\url{https://www.youtube.com/watch?v=ovJcsL7vyrk}
	\item\url{https://www.chiark.greenend.org.uk/~pcorbett/logistichisto.html}


\end{itemize}



\end{document}
